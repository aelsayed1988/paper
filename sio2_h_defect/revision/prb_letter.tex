\documentclass[10pt,a4paper,]{moderncv}        % possible options include font size ('10pt', '11pt' and '12pt'), paper size ('a4paper', 'letterpaper', 'a5paper', 'legalpaper', 'executivepaper' and 'landscape') and font family ('sans' and 'roman')
%\usepackage{enumitem}

% moderncv themes
\moderncvstyle{classic}                             % style options are 'casual' (default), 'classic', 'oldstyle' and 'banking'
\moderncvcolor{blue}                               % color options 'blue' (default), 'orange', 'green', 'red', 'purple', 'grey' and 'black'
%\renewcommand{\familydefault}{\sfdefault}         % to set the default font; use '\sfdefault' for the default sans serif font, '\rmdefault' for the default roman one, or any tex font name
%\nopagenumbers{}                                  % uncomment to suppress automatic page numbering for CVs longer than one page

% character encoding
%\usepackage[utf8]{inputenc}                       % if you are not using xelatex ou lualatex, replace by the encoding you are using
%\usepackage{CJKutf8}                              % if you need to use CJK to typeset your resume in Chinese, Japanese or Korean

% adjust the page margins
\usepackage[scale=0.8]{geometry}
%\setlength{\hintscolumnwidth}{3cm}                % if you want to change the width of the column with the dates
%\setlength{\makecvtitlenamewidth}{10cm}           % for the 'classic' style, if you want to force the width allocated to your name and avoid line breaks. be careful though, the length is normally calculated to avoid any overlap with your personal info; use this at your own typographical risks...

% personal data
%%-----       letter       ---------------------------------------------------------
\name{Al-Moatasem}{\mbox{El-Sayed}}
%\title{\emph{Curriculum Vitae}}                               % optional, remove / comment the line if not wanted
\address{University College London}{Gower Street, London, WC1E 6BT}{United Kingdom}% optional, remove / comment the line if not wanted; the "postcode city" and "country" arguments can be omitted or provided empty
\phone[mobile]{+44 (0) 7983 577077}                   % optional, remove / comment the line if not wanted; the optional "type" of the phone can be "mobile" (default), "fixed" or "fax"
\phone[fixed]{+44 (0)20 7679 7875}
%\phone[fax]{+3~(456)~789~012}
%\email{al-moatasem.el-sayed.10@ucl.ac.uk}                               % optional, remove / comment the line if not wanted
\homepage{http://www.cmmp.ucl.ac.uk/~kpm/people/tasem.htm}                         % optional, remove / comment the line if not wanted
%\social[linkedin]{john.doe}                        % optional, remove / comment the line if not wanted
%\social[twitter]{jdoe}                             % optional, remove / comment the line if not wanted
%\social[github]{jdoe}                              % optional, remove / comment the line if not wanted
%\extrainfo{Full name: Al-Moatasem Bellah El-Sayed \\ Date of Birth: 19$^{th}$ January, 1988\\Nationality: British\\Student Number: ABELS38}                 % optional, remove / comment the line if not wanted
\begin{document}
\recipient{Physical Review B}{1 Research Road,\\ Ridge,\\ NY 11961-2701} 
\date{2$^{\textrm{nd}}$ April, 2015}
\opening{Dear Editor,} 
\closing{Yours faithfully,}
%\enclosure[Attached]{curriculum vit\ae{}}          % use an optional argument to use a string other than "Enclosure", or redefine \enclname
\makelettertitle

In this manuscript we address the important question of the interaction of atomic and molecular hydrogen with amorphous silica network. Atomic hydrogen is produced by photolysis or radiolysis of silica glass, during anneal and hole injection in CMOS devices, and plays an important role in tectosilicate minerals. It is usually assumed that atomic hydrogen interacts weakly with non-defective silica networks, instead reacting solely with structural defects. 

We show that the interaction of atomic hydrogen with strained \mbox{Si--O} bonds in defect-free a-SiO$_2$ networks results in the formation of two distinct defect structures referred to as the [SiO$_4$/H]$^0$ and the hydroxyl E$^\prime$ center. We study the distribution of each defect's properties and demonstrate that the hydroxyl E$^\prime$ center can be thermodynamically stable in the neutral charge state. We describe in detail the interaction of H with a single oxygen vacancy in a-SiO$_2$. In order to understand the origins and reactions of these defects, different mechanisms of formation, passivation and de-passivation have been investigated. 

These results provide a better understanding of how atomic and molecular hydrogen can both passivate existing defects and create new electrically active defects in amorphous silica matrices. Better understanding of the reactivity of atomic hydrogen may have significant implications for our understanding of processes in silica glass and nano-scaled silica, e.g. in porous low-permittivity insulators, and strained variants of a-SiO$_2$, and therefore should be of interest to a wide community of chemists, mineralogists, physicists and engineers, which comprise the PRB readership.

\makeletterclosing
%
%%\clearpage\end{CJK*}                              % if you are typesetting your resume in Chinese using CJK; the \clearpage is required for fancyhdr to work correctly with CJK, though it kills the page numbering by making \lastpage undefined
%
%
%%% end of file `template.tex'.
%%% start of file `template.tex'.
%%% Copyright 2006-2013 Xavier Danaux (xdanaux@gmail.com).
%%
%% This work may be distributed and/or modified under the
%% conditions of the LaTeX Project Public License version 1.3c,
%% available at http://www.latex-project.org/lppl/.
%
%\clearpage
%\photo[64pt][0.4pt]{picture}                       % optional, remove / comment the line if not wanted; '64pt' is the height the picture must be resized to, 0.4pt is the thickness of the frame around it (put it to 0pt for no frame) and 'picture' is the name of the picture file
%\quote{Some quote}                                 % optional, remove / comment the line if not wanted

% to show numerical labels in the bibliography (default is to show no labels); only useful if you make citations in your resume
%\makeatletter
%\renewcommand*{\bibliographyitemlabel}{\@biblabel{\arabic{enumiv}}}
%\makeatother
%\renewcommand*{\bibliographyitemlabel}{[\arabic{enumiv}]}% CONSIDER REPLACING THE ABOVE BY THIS

% bibliography with mutiple entries
%\usepackage{multibib}
%\newcites{book,misc}{{Books},{Others}}
%----------------------------------------------------------------------------------
%            content
%----------------------------------------------------------------------------------
%\begin{CJK*}{UTF8}{gbsn}                          % to typeset your resume in Chinese using CJK
%-----       resume       ---------------------------------------------------------
%\makecvtitle
%
%\section{Education}
%\cventry{September 2010--March 2015}{Ph.D; Condensed Matter Physics; Supervisor: Prof. Alexander Shluger}{University College London}{}{My main research interests are in the use of atomistic modelling techniques to understand chemical and physical properties of materials used in electronic devices. Electronic device reliability issues hinder progress in developing more powerful, reliable electonic devices. The use of theoretical modelling can complement and enlighten experimental analyses of the complex experimental data obtained from nanoscale devices}{}
%
%%My current research is focused on the physics and chemistry of defects in oxides and at semiconductor/oxide interfaces and understanding their effects on electronic devices. Oxides are widely used as dielectrics in microelectronic devices, such as the oxides in metal-oxide-semiconductor field effect transistors (MOSFETs) and resistive RAM technologies (ReRAM). Defects in oxides affect the electrical properties of the materials and are implicated in reliability issues in MOSFET devices and the fundamental operation of ReRAM devices. These defects and their reactions are difficult to resolve experimentally and the use of theoretical modelling can complement and enlighten experimental analyses of the complex experimental data resulting from defects.}{}  % arguments 3 to 6 can be left empty
%%\cventry{year--year}{Degree}{Institution}{City}{\textit{Grade}}{Description}
%\cventry{2006--2010}{M. Chem; Chemistry with Study in Europe}{University of Manchester}{}{\textit{2.1}}{}
%\cventry{1998--2006}{Secondary School and Sixth Form}{King Edward VII School}{Sheffield}{\emph{A-Levels}: Chemistry (A), Maths (A), Spanish (B), Biology (B); \emph{GCSEs}: Maths (A), Science (Dual Award AA), English (B), French (A), Spanish (A), Physical Education (B), Business Studies (B), Religious Education (Short course, A), Arabic (A*) }{}
%
%\section{Languages}
%\cvdoubleitem{Arabic}{Native}{Spanish}{Fluent}
%\cvdoubleitem{French}{Conversational}{Mandarin}{Beginner/Intermediate}
%
%\section{Computer Skills}
%\cvitem{Programming Languages}{FORTRAN, BASH, Python, Unix Shell Scripts, MPI Parallel Processing libraries}
%\cvitem{Applications}{Unix Mapping software, \LaTeX, Windows Office suites, JOOMLA \& Wordpress Management Software, UNIX Administration and Presentation software, Blender 3D Modelling, Molecular Modelling Software}
%
%\section{Interests}
%\cvitem{Motorbike and Cycle Enthusiast}{I enjoy riding bicycles and motorbikes and have extensive knowledge of both. I have rebuilt motorcycles and enjoy looking after and repairing both motorcycles and bicycles.}
%\cvitem{Technology}{I enjoy using advanced technologies to make small trinkets. For example, I used 3D modelling software to design a small boat and tortoises and printed them via a 3D printer. This has also been useful in my academic career as I printed 3D models of atomic structures to develop more interactive presentations of my research.}
%%\cvitem{Conference organisation} {.}
%\cvitem {Work Social Events} {I am an active member of the Thomas Young centre, a consortium of materials modelling research groups across London, helping to organise social events for the group. As part of a team working across universities in London, I helped organise the Hermes conference  \url{http://www.hermes2012.org/}, focusing on material modelling and science communication. My role was to oversee the science communication aspect, collaborating with science communication students as well as science journalists from the BBC.}
%
%\renewcommand{\refname}{Selected Publications}
%\nocite{*}
%\bibliographystyle{plain}
%\bibliography{publications}                        % 'publications' is the name of a BibTeX file
%
%\section{Selected Conference Presentations}
%\cvitem{}{El-Sayed A.,Ling S., Watkins M. B., Shluger A. L., Nature of Intrinsic and Extrinsic Electron Traps in Silicon Dioxide, Towards Reality in Nano-Materials, Levi, Finland, 2014 }
%\cvitem{}{El-Sayed A.,Ling S., Gejo F. L., Watkins M. B., Shluger A. L., A Computational Study of Si-H Bonds as Precursors for Neutral E$^\prime$ centers in SiO$_{\textrm{2}}$, Insulating films on surfaces, Krakow, 2013 }
%%\item[] El-Sayed A., Watkins M., Shluger A., Electron trapping in Ge doped and bulk SiO$_{\textrm{2}}$, Thomas Young Centre, London, 2013 {\bf Talk}
%\cvitem{} {El-Sayed A., Gejo F. L., Watkins M., Shluger A.,Charged Defects in SiO$_{\textrm{2}}$ and their Relationship with Reliability Problems in MOSFETs, Center for Nanoelectronic Devices, Zewail City of Science and Technology, Cairo, Egypt, 2012} 
%%\cvitem{} {Gejo F. L., El-Sayed A., Shluger A.,The role of disorder in SiO$_{\textrm{2}}$, Workshop on Dielectrics in Microelectronics, Dresden, Germany, 2012 {\textbf Talk}}
%\cvitem{} {El-Sayed A., Gejo F. L., Watkins M., Shluger A., Electron trapping defects in bulk SiO$_{\textrm{2}}$, SiO$_{\textrm{2}}$ Advanced Dielectrics and Related Devices, Hyeres, France, 2012 }
%
%\section{Awards}
%\cvitem{Young Scientist Award}{Best presentation award for a talk entitled "Nature of Intrinsic and Extrinsic Electron Traps in Silicon Dioxide at the "SiO$_2$ - Advanced Dielectrics and Related Devices", held in Cagliari in June 2014.}
%\cvitem{Best Poster Award}{Award for the best poster for a poster entitled "Towards Understanding the Role of H-Related Defects in Electronic Device Reliability Issues" at the "EURODIM" conference, held in Canterbury in July 2014.}
%
%\section{References}
%
%\begin{cvcolumns}
%  \cvcolumn{Prof. Alexander L. Shluger}{\begin{itemize}\item[] Department of Physics and Astronomy,\item[] University College London,\item[] Gower Street,\item[] London,\item[] WC1E 6BT\item[] Tel: +44(0)2076769 1312\item[] E-mail: a.shluger@ucl.ac.uk\end{itemize}}
%  \cvcolumn{Dr. Keith P. McKenna}{\begin{itemize}\item[] Department of Physics, University of York,\item[] Heslington, \item[] York,\item[] YO10 5DD\item[] Tel: +44 (0) 1904 322251\item[] E-mail: keith.mckenna@york.ac.uk\end{itemize}}
%\end{cvcolumns}
%%(more upon request)
%
%%\end{document}
%
%% Publications from a BibTeX file using the multibib package
%%\section{Publications}
%%\nocitebook{book1,book2}
%%\bibliographystylebook{plain}
%%\bibliographybook{publications}                   % 'publications' is the name of a BibTeX file
%%\nocitemisc{misc1,misc2,misc3}
%%\bibliographystylemisc{plain}
%%\bibliographymisc{publications}                   % 'publications' is the name of a BibTeX file
\end{document}
